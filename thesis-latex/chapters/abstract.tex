\addcontentsline{toc}{chapter}{\numberline{}Abstract}
\centerline {\textbf {\LARGE ABSTRACT}}

\vspace*{1cm}
In the recent era of technology data is everything. One of the very useful type of data is spatial data. The information about spatial data can be found to be much useful in many fields of industry. From remote sensing, mining to doing spatio-temporal analysis to doing flood or disease spread analysis, the applications are endless. However spatial data is not generally publicly easy to find. Spatial data is not handled well by the general purpose search engines. The methods to access these data are also different and many times require licensing and usage of proprietary technologies. Heterogeneity of the data poses another problem of ranking and combining.  A central catalog service can be built to keep track of this various repositories, also the data can be made available through simple unified calling mechanisms. With the availability of large amount of heterogeneous data from different multiple sources many type of operations like spatio temporal analysis can be made possible. This catalog service can then behave as central node for all available geo-spatial data available in the web. Registry can be used to publish all the information to subscribers and open web. Using this registry an efficient query processing system can be built around it to generate information needed in the user friendly mode. Query processor can find data from various heterogeneous data stores and process the data to get one result from many data sources. Various ranking and cost matrices can be deployed to make indexing of returned data feasible.\\
%\hspace*{1cm}\indent 
% \vspace*{1cm}
% The spread of incorrect information on Twitter can lead to harmful effect on individuals as people may consume and derive wrong inference from the information that reaches to them.There is thus a need to design a framework to limit the spread of such information.This thesis develops a model by harnessing the power of information retrieval algorithms to detect rumors.The tweets are collected and clustered using semantic information extracted from them.Each cluster has unique properties which can facilitate in the decision to consider it as rumor or non rumor.The work done extracts temporal properties from the clustered tweets and uses it to categorize rumors.Non Temporal properties are also considered for evaluation and they also play an important role in rumor detection.This ability to track the rumors has many real-time applications in various domains and ultimately systems can be designed to react to rumors once they are detected using such a framework. 

\begin{flushleft}
% {\textbf{Key words}: Rumor Propagation, Information Retrieval,Tweet Clustering,Semantic Information,Temporal Characteristics,Rumor Detection}\\
{\textbf{Key words}: Topical Crawling, Meta-Data Discovery, Meta-Data Publishing, Spatial Ranking, Spatial Cloud Computing, Spatial Query Orchestration}\\

\end{flushleft}







