% Page 11 : Abstract
\thispagestyle{empty}
\begin{center}
\textbf{\Large Abstract}
\end{center}
In the recent era of technology data is everything. One of the very useful type of data is spatial data. The information about spatial data can be found to be much useful in many fields of industry. From remote sensing, mining to doing spatio-temporal analysis to doing flood or disease spread analysis, the applications are endless. However this data is not generally publicly easy to find. Spatial data is not handled well by the the general purpose search engines. The methods to access these data are also heterogeneous and many times require licensing and using proprietary technologies. Heterogeneity of the data poses another problem of ranking and combining  A central catalog service can be built to keep track of this various repositories, also the data can be made available through simple unified calling mechanisms. With the availability of large amount of heterogeneous data from different multiple sources one can do many type of spatio temporal analysis. This catalog service can then behave as central node for all available geo-spatial data available in the web. Registry can be used to publish all the information to subscribers and open web. Using this registry an efficient query processing system can be built around it to generate information needed in the user friendly mode. Query processor can find data from various heterogeneous data stores and process the data to get one result from many data sources. Various ranking and cost matrices can be deployed to make indexing of returned data feasible.
\newpage
\thispagestyle{empty}
%\clearpage
%\cleardoublepage